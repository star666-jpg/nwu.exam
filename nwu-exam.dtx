% \iffalse meta-comment
%
% Copyright (C) 2025 by NWU-Exam Team
%
% \fi
%
% \iffalse
%<*driver>
\documentclass{l3doc}
\usepackage[UTF8, scheme=plain]{ctex} 
\usepackage{tikz}
\usepackage{eso-pic}
\usepackage{geometry}
\usepackage{amsmath}
\usepackage{fancyhdr}
\usepackage{lastpage}
\usepackage{chngcntr} % 用于计数器重置
\geometry{a4paper, left=3cm, right=2cm, top=3cm, bottom=3cm} 

\EnableCrossrefs
\CodelineIndex
\RecordChanges

\begin{document}
	\DocInput{nwu-exam.dtx}
\end{document}
%</driver>
% \fi
%
% \title{NWU Exam Class (V7.0 Final)}
% \author{NWU-Exam Team}
% \date{\today}
% \maketitle
%
% \section{核心实现}
%
%    \begin{macrocode}
%<*class>
\NeedsTeXFormat{LaTeX2e}
\RequirePackage{expl3}
\ProvidesExplClass{nwu-exam}{2025/06/01}{0.7}{NWU Exam Class Final}

% =======================================================
% 1. 基础宏包加载
% =======================================================
\LoadClass[a4paper,11pt]{article}
\RequirePackage{geometry}
\RequirePackage[scheme=plain]{ctex}
\RequirePackage{eso-pic}
\RequirePackage{tikz}
\RequirePackage{fancyhdr}
\RequirePackage{amsmath, amssymb}
\RequirePackage{array}
\RequirePackage{lastpage}
\RequirePackage{chngcntr} % 关键:用于大题重新计数

% 页面几何设置:底部留出空间给页脚
\geometry{left=2.5cm, right=2cm, top=2.5cm, bottom=3.0cm}

% =======================================================
% 2. 核心逻辑层 (全 Expl3 环境,抗干扰)
% =======================================================
\ExplSyntaxOn

% --- 变量定义 ---
\tl_new:N \g_nwu_term_tl       
\tl_new:N \g_nwu_subject_tl    
\tl_new:N \g_nwu_paper_type_tl 
\tl_new:N \g_nwu_department_tl 
\tl_new:N \g_nwu_date_tl       % 新增:考试日期

\NewDocumentCommand{\SetTerm}{m}{ \tl_gset:Nn \g_nwu_term_tl {#1} }
\NewDocumentCommand{\SetSubject}{m}{ \tl_gset:Nn \g_nwu_subject_tl {#1} }
\NewDocumentCommand{\SetPaperType}{m}{ \tl_gset:Nn \g_nwu_paper_type_tl {#1} }
\NewDocumentCommand{\SetDept}{m}{ \tl_gset:Nn \g_nwu_department_tl {#1} }
\NewDocumentCommand{\SetDate}{m}{ \tl_gset:Nn \g_nwu_date_tl {#1} } % 设置日期接口

% 默认值
\SetTerm{2021 ---- 2022~学年第一学期}
\SetSubject{数学分析}
\SetPaperType{A}
\SetDept{数学学院}
\SetDate{\the\year 年\the\month 月\the\day 日} % 默认当天

% --- 2.1 页眉生成 (Header) - 修正总分栏 ---
\NewDocumentCommand{\MakePaperHeader}{ }{
	\begin{center}
		% 标题
		{ \zihao{2} \heiti \bfseries \g_nwu_term_tl } \\
		\vspace{0.8cm}
		
		% 表格:改为4列结构,给总分留出填写框
		\renewcommand{\arraystretch}{1.6}
		\begin{tabular}{| p{2.5cm} | p{8cm} | p{1.5cm} | p{2cm} |}
			\hline
			\centering \textbf{考试科目} & \centering \g_nwu_subject_tl & \centering \textbf{总分} &  \\
			\hline
		\end{tabular}
	\end{center}
	
	\vspace{0.3cm}
	% 警示语
	\noindent \textbf{\underline{注意:答案一律写在答题纸上,否则无效!}}
	\par \vspace{0.5cm}
}

% --- 2.2 页脚表格 (Footer) - 改为单行 ---
\pagestyle{fancy}
\fancyhf{} 
\renewcommand{\headrulewidth}{0pt}

\fancyfoot[C]{
	\begin{tikzpicture}[remember~picture, overlay]
	% 定位:页面底部向上 1.5cm
	\node[yshift=1.5cm] at (current~page.south) {
		\footnotesize
		\renewcommand{\arraystretch}{1.3}
		% 单行表格设计
		\begin{tabular}{|c|c|c|c|c|c|c|c|c|c|}
		\hline
		本卷为 & 团卷 & 本卷为 & \g_nwu_paper_type_tl ~卷 & 印数 & 220 & 出题院系 & \g_nwu_department_tl & 出题人 & \underline{\hspace{2em}} \\
		\hline
		\end{tabular}
	};
	% 页码:显示在表格下方
	\node[yshift=0.8cm] at (current~page.south) { \thepage \ / \pageref{LastPage} };
	\end{tikzpicture}
}

% --- 2.3 密封线 (Sealed Line) ---
\newcommand{\DrawSealedLine}{%
	\AddToShipoutPictureBG{%
		\begin{tikzpicture}[remember~picture, overlay]
		\tikzset{sealed/.style={dashed, color=black!60, line~width=0.8pt}}
		\tikzset{textstyle/.style={rotate=90, anchor=center, color=black}}
		
		\draw[sealed] ([xshift=1.5cm]current~page.south~west) -- ([xshift=1.5cm]current~page.north~west);
		\draw[sealed] ([xshift=1.8cm]current~page.south~west) -- ([xshift=1.8cm]current~page.north~west);
		
		\node[textstyle] at ([xshift=1.65cm]current~page.west) {
			\zihao{4}\heiti\bfseries \quad 密 \qquad 封 \qquad 线 \quad
		};
		
		\node[textstyle, anchor=east] at ([xshift=1.3cm, yshift=6cm]current~page.west) {
			\zihao{5}\songti 姓~~名:\underline{\makebox[3cm]{}}
		};
		\node[textstyle, anchor=east] at ([xshift=1.3cm, yshift=0cm]current~page.west) {
			\zihao{5}\songti 学~~号:\underline{\makebox[3cm]{}}
		};
		\node[textstyle, anchor=east] at ([xshift=1.3cm, yshift=-6cm]current~page.west) {
			\zihao{5}\songti 专~~业:\underline{\makebox[3cm]{}}
		};
		\end{tikzpicture}%
	}%
}
\AtBeginDocument{\DrawSealedLine}

% --- 2.4 智能选择题逻辑 ---
\dim_new:N \l_nwu_choice_max_width_dim
\dim_new:N \l_nwu_line_width_dim
\box_new:N \l_nwu_choice_a_box
\box_new:N \l_nwu_choice_b_box
\box_new:N \l_nwu_choice_c_box
\box_new:N \l_nwu_choice_d_box

\cs_new_protected:Npn \__nwu_print_item:nnn #1#2#3 {
	\dim_compare:nNnTF { #3 } = { 0pt }
	{ \par \noindent \hspace*{2em} \makebox[2em][l]{\bfseries #1.} \parbox[t]{ \linewidth - 4.5em } { #2 } }
	{ \makebox[#3][l]{ \hspace*{2em} \makebox[1.5em][l]{\bfseries #1.} #2 } }
}

\cs_new_protected:Npn \nwu_choices:nnnn #1#2#3#4 {
	\hbox_set:Nn \l_nwu_choice_a_box { #1 }
	\hbox_set:Nn \l_nwu_choice_b_box { #2 }
	\hbox_set:Nn \l_nwu_choice_c_box { #3 }
	\hbox_set:Nn \l_nwu_choice_d_box { #4 }
	
	\dim_set:Nn \l_nwu_choice_max_width_dim { 0pt }
	\dim_set:Nn \l_nwu_choice_max_width_dim { \dim_max:nn { \l_nwu_choice_max_width_dim } { \box_wd:N \l_nwu_choice_a_box } }
	\dim_set:Nn \l_nwu_choice_max_width_dim { \dim_max:nn { \l_nwu_choice_max_width_dim } { \box_wd:N \l_nwu_choice_b_box } }
	\dim_set:Nn \l_nwu_choice_max_width_dim { \dim_max:nn { \l_nwu_choice_max_width_dim } { \box_wd:N \l_nwu_choice_c_box } }
	\dim_set:Nn \l_nwu_choice_max_width_dim { \dim_max:nn { \l_nwu_choice_max_width_dim } { \box_wd:N \l_nwu_choice_d_box } }
	\dim_add:Nn \l_nwu_choice_max_width_dim { 4em }
	\dim_set:Nn \l_nwu_line_width_dim { \linewidth }
	
	\par \addvspace{0.5em} \noindent
	\dim_compare:nNnTF { 4 \l_nwu_choice_max_width_dim } < { \l_nwu_line_width_dim }
	{
		\__nwu_print_item:nnn {A} {#1} {0.25\linewidth}
		\__nwu_print_item:nnn {B} {#2} {0.25\linewidth}
		\__nwu_print_item:nnn {C} {#3} {0.25\linewidth}
		\__nwu_print_item:nnn {D} {#4} {0.25\linewidth}
	}
	{
		\dim_compare:nNnTF { 2 \l_nwu_choice_max_width_dim } < { \l_nwu_line_width_dim }
		{
			\__nwu_print_item:nnn {A} {#1} {0.5\linewidth}
			\__nwu_print_item:nnn {B} {#2} {0.5\linewidth} \\ \vspace{0.3em}
			\__nwu_print_item:nnn {C} {#3} {0.5\linewidth}
			\__nwu_print_item:nnn {D} {#4} {0.5\linewidth}
		}
		{
			\__nwu_print_item:nnn {A} {#1} {0pt}
			\__nwu_print_item:nnn {B} {#2} {0pt}
			\__nwu_print_item:nnn {C} {#3} {0pt}
			\__nwu_print_item:nnn {D} {#4} {0pt}
		}
	}
	\par \addvspace{0.5em}
}
\NewDocumentCommand{\xx}{ m m m m }{ \nwu_choices:nnnn {#1} {#2} {#3} {#4} }

% --- 2.5 题目与计分逻辑 ---
\newcounter{question} 
\newcounter{nwu_part_cnt}[question] 
\fp_new:N \g_nwu_total_points_fp 
\fp_new:N \l_nwu_current_points_fp
\seq_new:N \g_nwu_score_seq 
\bool_new:N \g_nwu_show_answer_bool
\bool_set_false:N \g_nwu_show_answer_bool 

\NewDocumentCommand{\fillin}{ O{} O{3cm} }
{
	\underline{ \makebox[#2][c]{ \bool_if:NTF \g_nwu_show_answer_bool { \bfseries #1 } { } } }
	\hspace{0.2em}
}

\NewDocumentCommand{\tf}{ O{} }{
	\unskip \nobreak \dotfill 
	( ~ \makebox[1.5em][c]{
		\bool_if:NTF \g_nwu_show_answer_bool
		{ \str_case:nnF { #1 } { {T} { $\checkmark$ } {F} { $\times$ } } { #1 } }
		{ } 
	} ~ )
}

\NewDocumentCommand{\sol}{ }{ \par \noindent \textbf{解:} }
\NewDocumentCommand{\zm}{ }{ \par \noindent \textbf{证明:} }

\NewDocumentEnvironment{question}{ O{0} }
{
	\refstepcounter{question}
	\fp_set:Nn \l_nwu_current_points_fp { #1 }
	\fp_gadd:Nn \g_nwu_total_points_fp { \l_nwu_current_points_fp }
	
	\fp_compare:nNnTF { \l_nwu_current_points_fp } > { 0 }
	{ \seq_gput_right:Ne \g_nwu_score_seq { \fp_to_decimal:N \l_nwu_current_points_fp } }
	{ \seq_gput_right:Nn \g_nwu_score_seq { - } }
	
	\par \vspace{0.5em} \noindent
	\textbf{\thequestion .}~
	\ignorespaces
}
{ \par \vspace{0.5em} }

\NewDocumentCommand{\qpart}{ O{0} }
{
	\stepcounter{nwu_part_cnt}
	\par \vspace{0.3em} \hspace{2em} 
	(\alph{nwu_part_cnt})~
	\fp_set:Nn \l_tmpa_fp { #1 }
	\fp_compare:nNnTF { \l_tmpa_fp } > { 0 }
	{
		\textbf{ ( \fp_to_decimal:N \l_tmpa_fp ~ 分 ) } ~
		\fp_gadd:Nn \g_nwu_total_points_fp { \l_tmpa_fp }
	}
	{ }
}

\NewDocumentCommand{\MakeGradingTable}{ }{
	\par \vspace{2em}
	\begin{center}
		\renewcommand{\arraystretch}{1.5}
		\setlength{\tabcolsep}{10pt}
		\begin{tabular}{ | c | *{\seq_count:N \g_nwu_score_seq}{c|} c | }
			\hline
			\textbf{题号} 
			\int_step_inline:nn { \seq_count:N \g_nwu_score_seq } { & \textbf{##1} }
			& \textbf{总分} \\
			\hline
			\textbf{分值}
			\seq_map_inline:Nn \g_nwu_score_seq { & ##1 }
			& \fp_to_decimal:N \g_nwu_total_points_fp \\
			\hline
			\textbf{得分}
			\int_step_inline:nn { \seq_count:N \g_nwu_score_seq } { & }
			& \\
			\hline
		\end{tabular}
	\end{center}
}

\ExplSyntaxOff

% =======================================================
% 3. 全局设置 (标题与编号)
% =======================================================
\ctexset{
	section = {
		number = \chinese{section},
		format = \zihao{4}\bfseries\heiti,
		name = { ,、},
		aftername = {},
		indent = 0pt,
		beforeskip = 1ex,
		afterskip = 1ex
	}
}

% 关键逻辑:每遇到 \section 就重置 question 计数器
\counterwithin*{question}{section}

%</class>
%    \end{macrocode}