% \iffalse meta-comment
%
% Copyright (C) 2025 by NWU-Exam Team
%
% \fi
%
% \iffalse
%<*driver>
\documentclass{l3doc}
\usepackage[UTF8, scheme=plain]{ctex} 
\usepackage{tikz}
\usepackage{eso-pic}
\usepackage{geometry}
\geometry{a4paper, left=3cm} 

\EnableCrossrefs
\CodelineIndex
\RecordChanges

\begin{document}
	\DocInput{nwu-exam.dtx}
\end{document}
%</driver>
% \fi
%
% \title{The \textsf{nwu-exam} class \\ 西北大学试卷排版系统}
% \author{NWU-Exam Team}
% \date{\today}
% \maketitle
%
% \section{核心实现}
%
%    \begin{macrocode}
%<*class>
\NeedsTeXFormat{LaTeX2e}
\RequirePackage{expl3}
\ProvidesExplClass{nwu-exam}{2025/06/01}{0.1}{NWU Exam Class}

% 1. 基础模块
\LoadClass[a4paper,11pt]{article}
\RequirePackage{geometry}
\RequirePackage[scheme=plain]{ctex} 
\RequirePackage{eso-pic}
\RequirePackage{tikz}

% 关闭 expl3 以适配 TikZ
\ExplSyntaxOff 
\newcommand{\DrawSealedLine}{%
	\AddToShipoutPictureBG{%
		\begin{tikzpicture}[remember picture, overlay]
		\tikzset{sealed/.style={dashed, color=black!60, line width=0.8pt}}
		\tikzset{textstyle/.style={rotate=90, anchor=center, color=black}}
		\draw[sealed] ([xshift=1.5cm]current page.south west) -- ([xshift=1.5cm]current page.north west);
		\draw[sealed] ([xshift=1.8cm]current page.south west) -- ([xshift=1.8cm]current page.north west);
		\node[textstyle] at ([xshift=1.65cm]current page.west) {
			\zihao{4}\heiti\bfseries \quad 密 \qquad 封 \qquad 线 \quad
		};
		\node[textstyle, anchor=east] at ([xshift=1.3cm, yshift=6cm]current page.west) {
			\zihao{5}\songti 姓~~名:\underline{\makebox[3cm]{}}
		};
		\node[textstyle, anchor=east] at ([xshift=1.3cm, yshift=0cm]current page.west) {
			\zihao{5}\songti 学~~号:\underline{\makebox[3cm]{}}
		};
		\node[textstyle, anchor=east] at ([xshift=1.3cm, yshift=-6cm]current page.west) {
			\zihao{5}\songti 专~~业:\underline{\makebox[3cm]{}}
		};
		\end{tikzpicture}%
	}%
}
\AtBeginDocument{\DrawSealedLine}
\ExplSyntaxOn
\msg_new:nnn { nwu-exam } { welcome } { Engine~Online. }

% =======================================================
% 3. 智能选择题模块 (Smart Choice Module) - V2.0 优化版
% =======================================================
\dim_new:N \l_nwu_choice_max_width_dim
\dim_new:N \l_nwu_line_width_dim
\box_new:N \l_nwu_choice_a_box
\box_new:N \l_nwu_choice_b_box
\box_new:N \l_nwu_choice_c_box
\box_new:N \l_nwu_choice_d_box

% 定义带缩进的选项打印函数 (Key Fix!)
% 参数 1: 标签 (A, B...)
% 参数 2: 内容
% 参数 3: 宽度 (0pt 表示自动流式布局,用于垂直模式)
\cs_new_protected:Npn \__nwu_print_item:nnn #1#2#3 {
	\dim_compare:nNnTF { #3 } = { 0pt }
	{ 
		% 垂直排列模式:使用 hanging indent
		% 模拟列表环境:标签宽 1.5em,标签与内容间距 0.5em
		\par
		\noindent
		\hspace*{2em} % 整体左缩进,对齐中文段落
		\makebox[2em][l]{\bfseries #1.} 
		\parbox[t]{ \linewidth - 4.5em } { #2 } 
		% 计算说明: 2em(整体缩进) + 2em(标签) + 0.5em(间隙) = 4.5em
	}
	{
		% 水平排列模式 (4列或2列)
		\makebox[#3][l]{ 
			\hspace*{2em} % 整体左缩进
			\makebox[1.5em][l]{\bfseries #1.} #2 
		}
	}
}

\cs_new_protected:Npn \nwu_choices:nnnn #1#2#3#4 {
	% 1. 测量
	\hbox_set:Nn \l_nwu_choice_a_box { #1 }
	\hbox_set:Nn \l_nwu_choice_b_box { #2 }
	\hbox_set:Nn \l_nwu_choice_c_box { #3 }
	\hbox_set:Nn \l_nwu_choice_d_box { #4 }
	
	% 2. 计算最大宽度
	\dim_set:Nn \l_nwu_choice_max_width_dim { 0pt }
	\dim_set:Nn \l_nwu_choice_max_width_dim { \dim_max:nn { \l_nwu_choice_max_width_dim } { \box_wd:N \l_nwu_choice_a_box } }
	\dim_set:Nn \l_nwu_choice_max_width_dim { \dim_max:nn { \l_nwu_choice_max_width_dim } { \box_wd:N \l_nwu_choice_b_box } }
	\dim_set:Nn \l_nwu_choice_max_width_dim { \dim_max:nn { \l_nwu_choice_max_width_dim } { \box_wd:N \l_nwu_choice_c_box } }
	\dim_set:Nn \l_nwu_choice_max_width_dim { \dim_max:nn { \l_nwu_choice_max_width_dim } { \box_wd:N \l_nwu_choice_d_box } }
	
	% 加上标签宽度修正 (2em indent + 1.5em label + 0.5em gap)
	\dim_add:Nn \l_nwu_choice_max_width_dim { 4em }
	
	\dim_set:Nn \l_nwu_line_width_dim { \linewidth }
	
	% 3. 布局决策
	\par \addvspace{0.5em} % 增加题干与选项的间距
	\noindent
	\dim_compare:nNnTF { 4 \l_nwu_choice_max_width_dim } < { \l_nwu_line_width_dim }
	{
		% Case A: 4列
		\__nwu_print_item:nnn {A} {#1} {0.25\linewidth}
		\__nwu_print_item:nnn {B} {#2} {0.25\linewidth}
		\__nwu_print_item:nnn {C} {#3} {0.25\linewidth}
		\__nwu_print_item:nnn {D} {#4} {0.25\linewidth}
	}
	{
		\dim_compare:nNnTF { 2 \l_nwu_choice_max_width_dim } < { \l_nwu_line_width_dim }
		{
			% Case B: 2列
			\__nwu_print_item:nnn {A} {#1} {0.5\linewidth}
			\__nwu_print_item:nnn {B} {#2} {0.5\linewidth} \\ \vspace{0.3em}
			\__nwu_print_item:nnn {A} {#3} {0.5\linewidth} % 注意: 这里应该是C
			\__nwu_print_item:nnn {B} {#4} {0.5\linewidth} % 注意: 这里应该是D
			% 修正标签:
		}
		{
			% Case C: 1列 (垂直)
			\__nwu_print_item:nnn {A} {#1} {0pt}
			\__nwu_print_item:nnn {B} {#2} {0pt}
			\__nwu_print_item:nnn {C} {#3} {0pt}
			\__nwu_print_item:nnn {D} {#4} {0pt}
		}
	}
	\par \addvspace{0.5em}
}

% 修正逻辑中的小笔误,并重新定义用户接口
\cs_set_protected:Npn \nwu_choices:nnnn #1#2#3#4 {
	% ... (测量代码同上,略) ...
	\hbox_set:Nn \l_nwu_choice_a_box { #1 }
	\hbox_set:Nn \l_nwu_choice_b_box { #2 }
	\hbox_set:Nn \l_nwu_choice_c_box { #3 }
	\hbox_set:Nn \l_nwu_choice_d_box { #4 }
	\dim_set:Nn \l_nwu_choice_max_width_dim { 0pt }
	\dim_set:Nn \l_nwu_choice_max_width_dim { \dim_max:nn { \l_nwu_choice_max_width_dim } { \box_wd:N \l_nwu_choice_a_box } }
	\dim_set:Nn \l_nwu_choice_max_width_dim { \dim_max:nn { \l_nwu_choice_max_width_dim } { \box_wd:N \l_nwu_choice_b_box } }
	\dim_set:Nn \l_nwu_choice_max_width_dim { \dim_max:nn { \l_nwu_choice_max_width_dim } { \box_wd:N \l_nwu_choice_c_box } }
	\dim_set:Nn \l_nwu_choice_max_width_dim { \dim_max:nn { \l_nwu_choice_max_width_dim } { \box_wd:N \l_nwu_choice_d_box } }
	\dim_add:Nn \l_nwu_choice_max_width_dim { 4em }
	\dim_set:Nn \l_nwu_line_width_dim { \linewidth }
	
	\par \addvspace{0.5em}
	\noindent
	\dim_compare:nNnTF { 4 \l_nwu_choice_max_width_dim } < { \l_nwu_line_width_dim }
	{
		\__nwu_print_item:nnn {A} {#1} {0.25\linewidth}
		\__nwu_print_item:nnn {B} {#2} {0.25\linewidth}
		\__nwu_print_item:nnn {C} {#3} {0.25\linewidth}
		\__nwu_print_item:nnn {D} {#4} {0.25\linewidth}
	}
	{
		\dim_compare:nNnTF { 2 \l_nwu_choice_max_width_dim } < { \l_nwu_line_width_dim }
		{
			\__nwu_print_item:nnn {A} {#1} {0.5\linewidth}
			\__nwu_print_item:nnn {B} {#2} {0.5\linewidth} \\ \vspace{0.3em}
			\__nwu_print_item:nnn {C} {#3} {0.5\linewidth}
			\__nwu_print_item:nnn {D} {#4} {0.5\linewidth}
		}
		{
			\__nwu_print_item:nnn {A} {#1} {0pt}
			\__nwu_print_item:nnn {B} {#2} {0pt}
			\__nwu_print_item:nnn {C} {#3} {0pt}
			\__nwu_print_item:nnn {D} {#4} {0pt}
		}
	}
	\par \addvspace{0.5em}
}

\NewDocumentCommand{\xx}{ m m m m }{
	\nwu_choices:nnnn {#1} {#2} {#3} {#4}
}% =======================================================
% =======================================================
% =======================================================
% 4. 题目与评分模块 (Question & Grading Module) - V2.0
% =======================================================

% --- 变量定义 ---
\newcounter{question} 
\newcounter{nwu_part_cnt}[question] 
\fp_new:N \g_nwu_total_points_fp 
\fp_new:N \l_nwu_current_points_fp

% 新增:分数序列 (用于存储每道题的分值)
\seq_new:N \g_nwu_score_seq 

% --- 题目环境 (question) ---
\NewDocumentEnvironment{question}{ O{0} }
{
	\refstepcounter{question}
	\fp_set:Nn \l_nwu_current_points_fp { #1 }
	\fp_gadd:Nn \g_nwu_total_points_fp { \l_nwu_current_points_fp }
	
	% 新增:将当前分数写入序列 (记录下来!)
	% 如果是0分,为了表格好看,我们也记录为 "-" 或 "0"
	\fp_compare:nNnTF { \l_nwu_current_points_fp } > { 0 }
	{ \seq_gput_right:Ne \g_nwu_score_seq { \fp_to_decimal:N \l_nwu_current_points_fp } }
	{ \seq_gput_right:Nn \g_nwu_score_seq { - } }
	
	% 排版逻辑
	\par \vspace{1em} 
	\noindent
	\textbf{\thequestion .}~
	\fp_compare:nNnTF { \l_nwu_current_points_fp } > { 0 }
	{ \textbf{ ( \fp_to_decimal:N \l_nwu_current_points_fp ~ 分 ) } ~ }
	{ } 
	\ignorespaces
}
{ \par }

% --- 小题环境 ---
\NewDocumentCommand{\qpart}{ }{
	\stepcounter{nwu_part_cnt}
	\par \vspace{0.3em}
	\hspace{2em} (\alph{nwu_part_cnt})~
}

% --- 动态得分表生成引擎 (核心算法) ---
\NewDocumentCommand{\MakeGradingTable}{ }{
	\par \vspace{2em}
	\begin{center}
		% 使用简单的 array 语法,动态计算列数
		% 结构:| 题号 | 1 | 2 | ... | 总分 |
		\renewcommand{\arraystretch}{1.5} % 表格行高
		\setlength{\tabcolsep}{10pt}      % 列宽宽松点
		
		% 开始绘制表格
		% 注意:这里用了一个技巧,先计算有多少道题,生成对应数量的 c|
		\begin{tabular}{ | c | *{\seq_count:N \g_nwu_score_seq}{c|} c | }
			\hline
			% 第一行:表头 (题号)
			\textbf{题号} 
			% 循环输出 1, 2, 3 ...
			\int_step_inline:nn { \seq_count:N \g_nwu_score_seq } { & \textbf{##1} }
			& \textbf{总分} \\
			\hline
			
			% 第二行:分值
			\textbf{分值}
			% 遍历序列,输出存储的分数
			\seq_map_inline:Nn \g_nwu_score_seq { & ##1 }
			& \fp_to_decimal:N \g_nwu_total_points_fp \\
			\hline
			
			% 第三行:得分 (留空给老师打分)
			\textbf{得分}
			\int_step_inline:nn { \seq_count:N \g_nwu_score_seq } { & }
			& \\
			\hline
		\end{tabular}
	\end{center}
}
%</class>
%    \end{macrocode}