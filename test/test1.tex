%!TeX program = xelatex
\documentclass{../nwu-exam}

% ==========================================
% 1. 试卷信息配置 (Header Info)
% ==========================================
\SetTerm{西北大学 2021 ---- 2022 学年第一学期}
\SetSubject{数学分析}
\SetPaperType{B}       % 设置为 B 卷
\SetDept{数学学院}     % 出题院系

\begin{document}
	% 生成试卷表头
	\MakePaperHeader
	
	% ==========================================
	% 第一部分:填空题
	% ==========================================
	\section{填空题(每小题 5 分,共 30 分)}
	
	\begin{question}[5]
		计算积分:$\int_{0}^{1}dy\int_{y}^{\sqrt{y}}\frac{\sin x}{x}dx = $ \fillin[][3cm].
	\end{question}
	
	\begin{question}[5]
		设 $I(y)=\int_{0}^{y}\frac{\ln(1+xy)}{x}dx \quad (y>0)$,则 $I'(2) = $ \fillin[][3cm].
	\end{question}
	
	\begin{question}[5]
		设平面曲线 $L$ 为 $y=\sqrt{4-x^{2}}$,则 $\oint_{L}(x^{2}+y^{2}+1)ds = $ \fillin[][3cm].
	\end{question}
	
	\begin{question}[5]
		设 $\Sigma$ 为球面 $x^{2}+y^{2}+z^{2}=1$,则 $\oint_{\Sigma}\sin x\cdot \sin y\cdot \sin z \, dS = $ \fillin[][3cm].
	\end{question}
	
	\begin{question}[5]
		平面有界点集 $D$ 可求面积的充分必要条件是 \fillin[][6cm].
	\end{question}
	
	\begin{question}[5]
		设 $D=\{(x,y)|x^{2}+y^{2}\ge1\}$。当 $p$ \fillin[][2cm] 时,反常重积分 $\iint_{D}\frac{1}{(x^{2}+y^{2})^{p}}dxdy$ 收敛。
	\end{question}
	
	% ==========================================
	% 第二部分:计算题
	% ==========================================
	\section{计算题(每小题 10 分,共 50 分)}
	
	\begin{question}[10]
		求二重积分 $\iint_{D}\frac{1}{(x^{2}+y^{2})^{2}}dxdy$,其中 $D=\{(x,y)|x^{2}+y^{2}\le 2x, x\ge 1\}$。
		\sol
		\par \vspace{5cm} % 留出答题空间
	\end{question}
	
	\begin{question}[10]
		计算由曲面 $\frac{x^{2}}{a^{2}}+\frac{y^{2}}{b^{2}}+\frac{z^{2}}{c^{2}}=1$ 所围空间立体的体积。
		\sol
		\par \vspace{5cm}
	\end{question}
	
	\begin{question}[10]
		设 $L$ 是单位圆周 $x^{2}+y^{2}=1$(取逆时针方向),计算曲线积分:
		\[ \oint_{L}\frac{(x-y)dx+(x+4y)dy}{x^{2}+4y^{2}} \]
		\sol
		\par \vspace{5cm}
	\end{question}
	
	\begin{question}[10]
		设 $\Sigma$ 为曲面 $x^{2}+z^{2}=4-y$ 且 $y\ge 0$ 的那部分的上侧,计算曲面积分:
		\[ \iint_{\Sigma} yz \, dydz + (x^{2}+z^{2})y \, dzdx + xy \, dxdy \]
		\sol
		\par \vspace{5cm}
	\end{question}
	
	\begin{question}[10]
		设球面的半径为 $R$,球心在球面 $x^{2}+y^{2}+z^{2}=a^{2}$ 上。问 $R$ 当取何值时,该球面在球面 $x^{2}+y^{2}+z^{2}=a^{2}$ 内部的面积最大?并求最大面积。
		\sol
		\par \vspace{6cm}
	\end{question}
	
	% ==========================================
	% 第三部分:证明题
	% ==========================================
	\section{证明题(每小题 10 分,共 20 分)}
	
	\begin{question}[10]
		证明:$F(y)=\int_{0}^{+\infty}\frac{\sin xy}{x}dx$ 在 $(0,+\infty)$ 上不一致收敛,但在 $(0,+\infty)$ 上连续。
		\zm
		\par \vspace{6cm}
	\end{question}
	
	\begin{question}[10]
		证明:$\int_{0}^{+\infty}e^{-x^{2}}dx=\frac{\sqrt{\pi}}{2}$。
		\zm
		\par \vspace{6cm}
	\end{question}
	
	% ==========================================
	% 第四部分:选做题
	% ==========================================
	\section{选做题(附加题)}
	
	\begin{question}
		设 $P(x,y), Q(x,y)$ 在光滑曲线 $L$ 上连续,$K$ 为 $L$ 的长度,且 $M=\max_{(x,y)\in L}\sqrt{P^{2}+Q^{2}}$。
		\par
		(1) 证明: $\left|\int_{L}P(x,y)dx+Q(x,y)dy\right| \le M\cdot K$。
		\par
		(2) 证明: $\lim_{R\rightarrow+\infty}\int_{L_{R}}\frac{ydx-xdy}{(x^{2}+xy+y^{2})^{2}}=0$,其中 $L_{R}$ 为圆周 $x^{2}+y^{2}=R^{2}$。
		\zm
		\par \vspace{5cm}
	\end{question}
	
	\begin{question}
		设 $u(x,y)$ 具有二阶连续偏导数,$\frac{\partial u}{\partial n}$ 为沿外法线方向的方向导数。证明:$u$ 是调和函数的充分必要条件为对 $D$ 中任意光滑封闭曲线 $L$,有
		\[ \oint_{L}\frac{\partial u}{\partial n}ds = 0 \]
		\zm
		\par \vspace{5cm}
	\end{question}
	
	% 生成得分统计表
	\MakeGradingTable
	
\end{document}